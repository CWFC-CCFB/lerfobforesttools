\documentclass[a4paper,12pt]{report}
%\usepackage[authoryear]{natbib}
\usepackage{jabes}
\usepackage{mathtools}
\usepackage{wrapfig}
\usepackage{fancyvrb}

\usepackage{color}
\def\r#1{\textcolor{blue}{\bf[#1]}}
\newcounter{interrogation}
\def\q#1{\refstepcounter{interrogation}\textcolor{red}{\bf[Q\arabic{interrogation}. #1]}}

\def\vect#1{\bm{#1}}
\def\matx#1{\mathbf{#1}}

\newcommand{\citetapos}[1]{\citeauthor{#1}'s \citeyearpar{#1}}

\newcommand{\authorlist}{Fortin, M.}
\newcommand{\runninghead}{CAT User's guide}
\title{The Carbon Assessment Tool (CAT) \\ User's guide}
\author{Mathieu Fortin}
\date{\today}

\begin{document}
\maketitle
\tableofcontents

\listoffigures
\listoftables

\chapter{Introduction}

\chapter{Structure and features}

\section{Internal structure}

The carbon assessment tool (CAT) does not run any growth simulations. Like other similar tools, it retrieves growth predictions from existing models or yield tables. Its structures is flexible so that it adapts to the nature of the predictions, the minimum requirement being that at least commercial volume predictions are available. 

It is based on a main module which handles tasks such as
\begin{itemize}
\item retrieving growth predictions
\item retrieving dead, windfall and harvested volumes
\item assessing the carbon in different compartments 
\item displaying and exporting the simulation results 
\end{itemize}

In addition to this main module, two other modules define the settings of the simulation. The first one is the biomass parameters module, which makes it possible to define 

\begin{itemize}
\item the branch and root biomass expansion factors 
\item the basic wood density factors
\item the carbon content
\end{itemize}

The flux module, which is more complex, handles 

\begin{itemize}
\item the bucking of the harvested trees
\item the fate of windfall and dead trees
\item the production lines through which goes the logs
\item the recycling or disposal of harvested wood products (HWP) after the useful lifetime
\item the average lifetime of the dead biomass and the HWP
\item the substitution of fossil fuel emissions
\end{itemize}

Altogether, these three modules, the main, the biomass and the flux module, provide enough flexibility to simulate the carbon balance at the stand or regional level from the forest to the disposal of HWP. All calculations are made in Mg of carbon. 

CAT has been specially designed to avoid carbon leaks due to simplistic assumptions such as the instantaneous oxidation of HWP as soon as they are taken out of the forest or all HWP are burned after the useful lifetime. The flux module makes it possible to track the products during and after their useful lifetime and it ensures there is no loss or creation of matter all along the simulation.
 
Finally, the tool is also compatible with Monte Carlo techniques. If the growth simulations are stochastic, CAT can run as many carbon assessments than the number of realizations, and provide a partial confidence intervals on the simulation results.

\section{Features}

\subsection{Biomass}

CAT is flexible in the sense that it uses the interface concept in programming to take the best it can from growth simulations. The only requirement is that the commercial volume is provided. However, if your model provides more, such as dry biomass or even the carbon content of the whole tree, CAT can take it instead of default conversion factors. The way to make CAT aware that your growth predictions include more than just commercial volume is by implementing some interfaces as thoroughly explained in Section~\q{ajouter la section}.

Following the IPCC good practices (\q{add the reference}), CAT takes into account the carbon in the living biomass. It distinguishes aboveground from belowground, which are respectively estimated as follows 

\begin{equation}
\hat{c}_{ijks} = v_{ijks} \cdot \text{bd}_s \cdot \text{bef}_{ag,s} \cdot \text{cc}_s 
\label{equationAboveGroundBiomassCarbon}
\end{equation}

\begin{equation}
\hat{c}_{ijks} = v_{ijks} \cdot \text{bd}_s \cdot \text{bef}_{bg,s} \cdot \text{cc}_s 
\label{equationBelowGroundBiomassCarbon}
\end{equation}

where $\hat{c}_{ijk}$ is the estimated carbon content of tree $j$ of species $s$ in plot $i$ at time $k$, $v_{ijk}$ is the commercial volume of the same tree (m$^3$), $\text{bd}_s$ is the average basic density factor of species $s$ (Mg m$^{-3}$ of dry biomass), $\text{bef}_{ag,s}$ and $\text{bef}_{bg,s}$ are the average aboveground and belowground biomass expansion factors respectively, and $\text{cc}_s$ is the carbon content.   

CAT already provides the IPCC default values as well as French default values to fit into Eq.~\ref{equationAboveGroundBiomassCarbon} and \ref{equationBelowGroundBiomassCarbon} for $\text{bd}_s$, $\text{bef}_{ag,s}$ and $\text{bef}_{bg,s}$. 
Whenever the model provides for some of these factors, the user can decide whether or not these model-specific values should be used.




It also considers the litter and the dead wood. However, soil carbon is assumed to be constant which is the IPCC default assumption. \q{David}'s review on the topic leads to the conclusion that this assumption is valid under traditional forest management. However, an intensive management with fine woody debris removal may cause a decrease of soil carbon stocks, and this effect is not taken into account in the current version of CAT.

\begin{equation}
\hat{c}_{ijks} = v_{ijks} \cdot \text{bd}_s \cdot \text{bef}_s \cdot \text{cc}_s 
\label{equationBiomassCarbon}
\end{equation}



\subsection{Harvested wood products (HWP)}

\subsection{Substitution effets and other fluxes}

\chapter{Java implementation}

\section{A Java library}

The carbon assessment tool is part of a Java library called lerfob-foresttools.jar. This library relies on the JFreeChart library for graphical purposes and on the REpicea project (\q{give the website}) for flux handling, statistical calculation, serialization, import/export and user interface features. 

CAT's user interface (UI) is fully independent from the platform on which the growth simulations are run as long as the JFreeChart library can be referenced. If CAT is run without using the UI, then the JFreeChart library is no longer required. In any context, the repicea.jar library from REpicea project is mandatory. 

The main class of 


\section{Using CAT out of CAPSIS framework}

\chapter{User interface}

\section{Main interface}

\section{Setting the biomass parameters}

\section{Setting the production lines and other fluxes}

\subsection{How the stems are bucked into logs}

\section{Running simulations}

\subsection{Simulation output}

\subsection{Comparing simulations}

\subsection{Exporting simulation results}

\end{document}